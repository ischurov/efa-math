\documentclass[a4paper,14pt]{report} 
\usepackage[T2A]{fontenc}
\usepackage[utf8]{inputenc} 
\usepackage[english,russian]{babel}
\usepackage{amssymb,amsfonts,amsmath,mathtext,cite,enumerate,float} 
\usepackage{amsthm,amsfonts,amsmath,amssymb,amscd} 
\usepackage[dvips]{graphicx} 
\graphicspath{{images/}}
\usepackage{cmap}
\usepackage{array}
\usepackage{longtable}
\usepackage[usenames]{color}
\usepackage{multirow,makecell,array}

\usepackage{smartdiagram}

\usepackage{indentfirst}
\sloppy					
\clubpenalty=10000		
\widowpenalty=10000
\definecolor{light-blue}{rgb}{0.8,0.85,1}
\definecolor{lgreen}{rgb}{0.9,1,0.8}
\usepackage{cite}
\usepackage[linktocpage=true,plainpages=false,pdfpagelabels=false]{hyperref}
\usepackage{tocloft}
\usepackage{tabularx}

\definecolor{linkcolor}{rgb}{0.9,0,0}
\definecolor{citecolor}{rgb}{0,0.6,0}
\definecolor{urlcolor}{rgb}{0,0,1}
\hypersetup{
    colorlinks, linkcolor={linkcolor},
    citecolor={citecolor}, urlcolor={urlcolor}
}

\graphicspath{{images-overview/}}

\graphicspath{{homeworkLatex/}}
\DeclareGraphicsExtensions{.pdf,.png,.jpg}
%\usepackage{showkeys}
\usepackage{setspace}
\frenchspacing
%\pagestyle {headings}
\DeclareMathOperator*{\tg1}{tg}

\makeatletter
\bibliographystyle{utf8gost705u}	
\renewcommand{\@biblabel}[1]{#1.} 
\makeatother

\usepackage{geometry} 
\geometry{left=2.5cm}
\geometry{right=2.5cm}
\geometry{top=2cm}
\geometry{bottom=2cm}

\begin{document}

\section*{ \fcolorbox{green}{lgreen}{Определение 1.1}}
\noindent
Пусть \bf D \rm -- набор независимых значений переменной $x$, \bf{E} \rm -- набор значений функции $f(x)$ (область определения функции).\\
%------------базовый уровень-------------
\bf Функция \rm -- это правило, по которому сопоставляется элемент $x$ из набора $D$ элементу $f(x)$ из набора $E$. \\
%------------промежуточный уровень------- (требует знания отображений)
\bf Функция \rm -- это однозначное отображение $x \in D$ в $f(x) \in E$.


\section*{ Пример 1.1}
Площадь круга ($A$) зависит от радиуса ($r$). $A$ задается уравнением $A = \pi r^2$. Каждому положительному значению $r$ соответствует только одно значение $A$, следовательно, говорим, что $A$ - функция от $r$.

\section*{ Пример 1.2}
Отправление посылки имеет цену ($C$), которая зависит от веса пакета ($w$). Однако нет простой формулы, показывающей связь между $w$ и $C$. Почтовое отделение вводит правило, по которому определяется цена $C$ в зависимости от веса $w$ посылки.

%-----------------------------------------
\section*{ Замечание 1.1}
Функцию $f(x)$ можно представить как черный ящик, где входной параметр есть $x$, а результат применения функции есть $f(x)$. 

\section*{ Замечание 1.2}
\bf Аргументом \rm функции $f(x)$ называют $x$ .

%-----------------------------------------
\section*{ \fcolorbox{green}{lgreen}{Определение 1.2}}
\bf Нулями \rm функции называют те значения аргумента $x$, в которых функция $f(x)$ принимает значение 0.

%-----------------------------------------
\section*{ \colorbox{light-blue}{Интерпретация Функций}}
\begin{itemize}
	\item вербально (путем словесного описания);
	\item численно (путем описания таблицы с численными значениями);
	\item визуально (путем построения графика);
	\item алгебраически (путем задания явной формулы).
\end{itemize}

Одна функция может быть представлена всеми способами. Таким образом, можно хорошо изучить функцию, переходя от одного представления к другому. Однако для некоторых функций сложились общепринятые способы представления.

\section*{ Пример 1.3}
Наиболее общий способ интерпретации функции, вычисляющей площадь круга, есть алгебраическая формула $A(r) = \pi r^2$. Так же функцию можно представить таблицей значений или графиком, но удобнее использовать формулу. Известно, что любая окружность имеет положительный радиус ${r|r>0} = (0,\inf)$, получаем, область определения функции $A(r)$ есть $(0,\inf)$.
\section*{ Пример 1.4}
\noindent
\begin{minipage}[t]{70mm}\parindent=2em
    \begin{tabular}{|c|c|}
\hline
$w$ (килограммы) & $C(w)$ (рубли) \\
\hline
$0 < w \le 1$ &	88\\

$1 < w \le 2$ &	105 \\

$2 < w \le 3$ & 122 \\

$3 < w \le 4$	& 139\\

$4 < w \le 5$ & 156\\

. & . \\

. & . \\

. & . \\
\hline
\end{tabular}
\end{minipage}
%\hfill
\begin{minipage}[h]{80mm}
 Опишем функцию словами. Пусть $C(w)$ - издержки перевозки посылки с весом $w$. В 2010 году Почта России установила тариф: 88 рублей за перевозку одного килограмма, за каждый дополнительный килограмм следует доплата 17 рублей. Однако одна посылка не может по весу превысить 13 килограмм. Таблица иллюстрирует тариф.
\end{minipage}
 
\section*{ Упражнение}
 Когда включаете кран с горячей водой, то температура $T$ воды зависит от того, как долго вода льется из крана. Нарисуйте график температуры $T$ как функцию от времени $T = f(t)$. Считать стартовым временем момент открытия крана. 
 
\subsection*{ Решение}
...
%-----------------------------------------

\section*{ Пример 1.n}
идет множество примеров с рисунками\\
(здесь я вижу, что должно быть что-то интерактивное)

%-----------------------------------------

\section*{ Упражнение}
идет множество задач\\
(задача, поле для ответа, если ответ неверный, то смотреть решение, если верный, то проверить свои рассуждения)

%-----------------------------------------

%------------базовый уровень-------------
\section*{\colorbox{light-blue}{Симметричные (четные) функции}}
\section*{ \fcolorbox{green}{lgreen}{Определение 1.3}}
\noindent
Функция называется \bf четной \rm, если она удовлетворяет условию $f(-x) = f(x)$ для любого значения $x$.
\section*{ Пример 1.6}
Функция $f(x) = x^2$ является четной, потому что $f(-x) = (-x)^2 = x^2 = f(x)$.

\section*{\colorbox{light-blue}{Нечетный функции}}
\section*{ \fcolorbox{green}{lgreen}{Определение 1.4}}
\noindent
Функция называется \bf нечетной \rm, если она удовлетворяет условию $f(-x) = -f(x)$ для любого значения $x$.
\section*{ Пример 1.7}
Функция $f(x) = x^3$ является нечетной, потому что $f(-x) = (-x)^3 = -x^3 = -f(x)$.

\section*{ Упражнение}
Определите четные и нечетные функции. Есть ли здесь функции, которые являются ни четными, ни нечетными?
\begin{enumerate}
	\item $f(x) = x^5 + x$
	\item $g(x) = 1 - x^4$
	\item $h(x) = 2x - x^2$
\end{enumerate}
\subsection*{ Решение}
...

%------------базовый уровень-------------

\section*{\colorbox{light-blue}{Возрастающие и убывающие функции}}
\section*{ \fcolorbox{green}{lgreen}{Определение 1.5}}
\noindent
Функция $f(x)$ называется \bf возрастающей \rm на интервале $I$, если выполняется неравенство \[ f(x_1) < f(x_2), \] \\ где для любой пары $x_1, x_2$ выполнятся $x_1 < x_2$ на интервале $I$.
\section*{ \fcolorbox{green}{lgreen}{Определение 1.6}}
\noindent
Функция $f(x)$ называется \bf убывающей \rm на интервале $I$, если выполняется неравенство \[ f(x_1) > f(x_2), \] \\ где для любой пары $x_1, x_2$ выполнятся  $x_1 < x_2$ на интервале $I$.
\section*{ \fcolorbox{green}{lgreen}{Определение 1.6}}
\noindent
Говорят, функция $f(x)$ \bf не убывает \rm на интервале $I$, если выполняется неравенство \[ f(x_1) \le f(x_2), \] \\ где для любой пары $x_1, x_2$ выполнятся $x_1 < x_2$ на интервале $I$.
\section*{ \fcolorbox{green}{lgreen}{Определение 1.5}}
\noindent
Говорят, функция $f(x)$ называется \bf не возрастает \rm на интервале $I$, если выполняется неравенство \[ f(x_1) \ge f(x_2), \] \\ где для любой пары $x_1, x_2$ выполнятся $x_1 < x_2$ на интервале $I$.

%-----------------------------------------
\section*{ Упражнение}
идет множество задач\\
(задача. поле для ответа. если ответ неверный, то смотреть решение, если верный, то проверить свои рассуждения)
%-----------------------------------------


\section*{ \colorbox{light-blue}{Функции в Математическом Моделировании}}
%\begin{itemize}
%	\item рациональные;
%	\item алгебраические;
%	\item тригонометрические;
%	\item экспоненциальные;
%	\item логарифмические.
%\end{itemize}

\colorbox{yellow}{\bf Что такое математическая модель?}\par
\vspace{.5cm}
Это математическое описание реального процесса. Например, изменение размера популяции, процесс поставки товаров, скорость падающего объекта, концентрация продукта в химической реакции, ожидание рождения ребенка, издержки сокращения выброса загрязняющих окружающих среду веществ. Главная задача математической модели есть описание поведения явления и дальнейшее предсказание действий объекта (объектов). В большинстве случаев для описания какого-либо процесса используют функции.\\

\smartdiagram[flow diagram:horizontal]{Реальный процесс (явление),
  Математическая модель, Математические выводы, Прогнозирование}
 
 
 \subsection*{Реальный процесс (явление) $\xrightarrow{\text{Формализация}}$ Математическая модель}
 
 Во-первых, необходимо определить процесс, который будет описываться в дальнейшем. После выбора процесса идет его изучение. На этапе изучения, определяются важные для описания свойства. Важно помнить, что существуют зависимые свойства, которые нужно описывать функциями, зависящими от нескольких аргументов (параметрические функции). Также необходимо понимать, что не все свойства явления должны быть описаны. Некоторые из них обычно опускаются для упрощения написания модели. Далее используются наколенные знания о явлении и математический аппарат, чтобы перевести явление физическое в явление математическое. Однако существуют ситуации, когда знаний о явлении недостаточно, чтобы формализовать процесс. Тогда используются эмпирически накопленные данные, собранные в таблицы. Данные обсчитываются на компьютере для построения графиков и выведения специальной формулы. \par
 
  \subsection*{Математическая модель $\xrightarrow{\text{Решение}}$ Математические выводы}

 
 Во-вторых, после описания явления через формулы необходимо найти решения модели, используя весь релевантный математический аппарат.
 
 \subsection*{Математические выводы $\xrightarrow{\text{Интерпертация}}$ Прогнозирование}
 В-третьих, все решения модели должны быть интерпретированы в рамках описания процесса, для того чтобы по ним было легко прогнозировать дальнейшее поведение явления.

\subsection*{Прогнозирование $\xrightarrow{\text{Тестирование}}$ Реальный процесс (явление)}
Наконец, каждый прогноз тестируется на эмпирических данных, для того чтобы проверить соответствие его действительности. Если модель предсказывает что-то нереальное, то необходимо вернуться на первый шаг и понять, в описании какого из свойств была допущена ошибка.\par
Важно отметить, что математическая модель никогда полностью не представляет явление или процесс. Хорошая модель описывает реальность достаточно, чтобы описать явление формулами, а также остается достаточно точной, чтобы применять модель для предсказания поведения процесса. Необходимо помнить, что модель имеет свои ограничения на параметры, переменные, использование и тд. \par
 
\section*{Различают следующие функции для математического моделирвоания}
\begin{itemize}
	\item линейные $y = f(x) = mx + b$;
	\item полиномиальные $P(x) = a_n x^n + a_{n-1} x^{n-1} + \dots + a_2 x^2 + a_1 x + a_0$;
	\item степенные $f(x) = x^{\alpha}$;
	\item рациональные $f(x) = \dfrac{P(x)}{Q(x)}$, где $P(x),Q(x)$ полиномиальные функции;
	\item алгебраические (функции, полученные комбинаций других функций путем сложением, умножением, вычитанием, делением, выделением корня);
	\item тригонометрические (например, $\sin x , \cos x, \tan x, \dots$ );
	\item экспоненциальные $f(x) = a^x$;
	\item логарифмические $f(x) = \log_a x$.
\end{itemize}


\section*{ \colorbox{light-blue}{Преобразования функций}}
\begin{itemize}
	\item $y = f(x) + c$;
	\item $y = f(x) - c$;
	\item $y = f(x + c)$;
	\item $y = f(x - c)$;

	\item $y = c f(x)$;
	\item $y = \frac{1}{c} f(x)$;
	\item $y = f(cx)$;
	\item $y = f(\frac{x}{c})$;
	\item $y = -f(x)$;
	\item $y = f(-x)$.
\end{itemize}

\section*{ \colorbox{light-blue}{Комбинации функций}}
\begin{itemize}
	\item сложение $h(x) = f(x) + g(x)$;
	\item вычитание $h(x) = f(x)- g(x)$;
	\item деление $h(x) = \dfrac{f(x)}{g(x)}$;
	\item умножение $h(x) = f(x) * g(x)$;
	\item обратная функция $h(x) = f(x)^{-1}$;
	\item композиция функций $(f \circ g)(x) = f(g(x))$.
\end{itemize}
\end{document}















