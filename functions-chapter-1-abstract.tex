\documentclass[a4paper,14pt]{report} 
\usepackage[T2A]{fontenc}
\usepackage[utf8]{inputenc} 
\usepackage[english,russian]{babel}
\usepackage{amssymb,amsfonts,amsmath,mathtext,cite,enumerate,float} 
\usepackage{amsthm,amsfonts,amsmath,amssymb,amscd} 
\usepackage[dvips]{graphicx} 
\graphicspath{{images/}}
\usepackage{cmap}
\usepackage{array}
\usepackage{longtable}
\usepackage[usenames]{color}
\usepackage{multirow,makecell,array}

\usepackage{indentfirst}
\sloppy					
\clubpenalty=10000		
\widowpenalty=10000

\usepackage{cite}
\usepackage[linktocpage=true,plainpages=false,pdfpagelabels=false]{hyperref}
\usepackage{tocloft}
\usepackage{tabularx}

\definecolor{linkcolor}{rgb}{0.9,0,0}
\definecolor{citecolor}{rgb}{0,0.6,0}
\definecolor{urlcolor}{rgb}{0,0,1}
\hypersetup{
    colorlinks, linkcolor={linkcolor},
    citecolor={citecolor}, urlcolor={urlcolor}
}

\graphicspath{{images-overview/}}

\graphicspath{{homeworkLatex/}}
\DeclareGraphicsExtensions{.pdf,.png,.jpg}
%\usepackage{showkeys}
\usepackage{setspace}
\frenchspacing
%\pagestyle {headings}
\DeclareMathOperator*{\tg1}{tg}

\makeatletter
\bibliographystyle{utf8gost705u}	
\renewcommand{\@biblabel}[1]{#1.} 
\makeatother

\usepackage{geometry} 
\geometry{left=2.5cm}
\geometry{right=2.5cm}
\geometry{top=2cm}
\geometry{bottom=2cm}

\begin{document}

\section*{ Определение 1.1}
\noindent
Пусть \bf{D} \rm -- набор независимых значений переменной $x$, \bf{E} \rm -- набор значений функции $f(x)$.\\
%------------базовый уровень-------------
\bf Функция \rm -- это правило, по которому сопоставляется элемент $x$ из набора $D$ элементу $f(x)$ из набора $E$. \\
%------------промежуточный уровень------- (требует знания отображений)
\bf Функция \rm -- это однозначное отображение $x \in D$ в $f(x) \in E$.


\section*{ Пример 1.1}
Площадь круга ($A$) зависит от радиуса ($r$). $A$ задается уравнением $A = \pi r^2$. Каждому положительному значению $r$ соответствует только одно значение $A$, следовательно, говорим, что $A$ - функция от $r$.

\section*{ Пример 1.2}
Отправление посылки имеет цену ($C$), которая зависит от веса пакета ($w$). Однако нет простой формулы, показывающей связь между $w$ и $C$. Почтовое отделение вводит правило, по которому определяется цена $C$ в зависимости от веса $w$ посылки.

%-----------------------------------------
\section*{ Замечание 1.1}
Функцию $f(x)$ можно представить как машину, где входной параметр это $x$, а результат применения функции есть $f(x)$. 

\section*{ Замечание 1.2}
\bf Аргументом \rm функции $f(x)$ называют $x$ .

%-----------------------------------------
\section*{ Определение 1.2}
\bf Нулями \rm функции называют те значения $x$, в которых $f(x)$ принимает значение 0.

%-----------------------------------------
\section*{ Интерпретация Функций}
\begin{itemize}
	\item вербально (путем словесного описания);
	\item численно (путем описания таблицы с численными значениями);
	\item визуально (путем построения графика);
	\item алгебраически (путем задания явной формулы).
\end{itemize}

%-----------------------------------------
\section*{ Пример 1.3}
идет множество примеров с рисунками\\
(здесь я вижу, что должно быть что-то интерактивное)

%-----------------------------------------
\section*{ Задача 1.1}
идет множество задач\\
(задача, поле для ответа, если ответ неверный, то смотреть решение, если верный, то проверить свои рассуждения)
%-----------------------------------------
\section*{ Функции в Математическом Моделировании}
\begin{itemize}
	\item рациональные;
	\item алгебраические;
	\item тригонометрические;
	\item экспоненциальные;
	\item логарифмические.
\end{itemize}

\section*{ Комбинации функций}
\begin{itemize}
	\item сложение;
	\item вычитание;
	\item деление;
	\item умножение;
	\item обратная функция.
\end{itemize}
\end{document}















