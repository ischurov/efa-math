\documentclass[11pt,a4paper]{article}
\usepackage[utf8]{inputenc}

\usepackage{amsmath,amssymb,amsthm}
\usepackage[mathscr]{eucal}
\usepackage[russian]{babel}
\usepackage{titlesec}
%\usepackage{mathtools}

% Поля и т.п.
\pagestyle{plain}
\usepackage[margin=2cm,centering]{geometry}

%==========================================================
% буквы для N,Z,R,C
\newcommand\NN{{\mathbb N}}
\newcommand\ZZ{{\mathbb Z}}
\newcommand\RR{{\mathbb R}}
\newcommand\CC{{\mathbb C}}

% Модуль и норма
\newcommand{\abs}[1]{\lvert #1\rvert}
\newcommand{\norm}[1]{\lVert #1\rVert}
\renewcommand*\thesubsection{\arabic{subsection}}
\renewcommand{\baselinestretch}{1.25}
% Если нужны подзаголовки, используйте команды \subsection и \subsubsection

\begin{document}
{\huge \bf{Числовые последовательности}}
\section{Определения}
    Пусть у нас имеется набор чисел $a_{i}\in \RR$ и $i=1,\,2,\,\dots$ - не обязательно конечный:\\ \begin{equation*}
        a_{1},\,a_{2},\,a_{3},\,a_{4},\,\dots\,,a_{n},\,\dots
    \end{equation*}
    
    Тогда говорят, что $a_{1}$ - первый член последовательности, $a_{2}$ - второй член последовательности, и в общем $a_{n}$ - $n$-ый член последовательности.
    
    Можно также обозначить последовательность $\left\{ a_{1},a_{2},a_{3},\dots\,\right\}$ как $\left\{a_{n}\right\}$ или $\left\{ a_{n}\right\} _{n=1}^{\infty}$
\subsection{Пример}
    Пусть последовательность задается формулой $a_{n}=\frac{n-1}{n}$.\\
    Тогда $a_{1}=\frac{1-1}{1}=0$, $a_{2}=\frac{2-1}{2}=\frac{1}{2}$, $a_{3}=\frac{3-1}{3}=\frac{2}{3}$ и так далее. В итоге имеем числовую последовательность:
    \begin{equation*}
        0,\,\frac{1}{2},\,\frac{2}{3},\,\frac{3}{4},\,\frac{4}{5},\,\dots
    \end{equation*}
\subsection{Пример}
    Дана последовательнось $\left\{ \frac{\left(-1\right)^{n}\left(n+1\right)}{3^{n}}\right\}$. Найти её первые 4 члена.
\subsection{Решение}
    Поскольку на $n$ никаких ограничений нет, то мы считаем, что $n=1,\,2,\,\dots$. В таком случае просто подставим в формулу для $a_{n}=\frac{\left(-1\right)^{n}\left(n+1\right)}{3^{n}}$ значения $n=1,\;n=2,\;n=3,\;n=4$:
    \begin{align*}
        a_{1}&=\frac{\left(-1\right)^{1}\left(1+1\right)}{3^{1}}=-\frac{2}{3} \\
        a_{2}&=\frac{\left(-1\right)^{2}\left(2+1\right)}{3^{2}}=\frac{3}{9}=\frac{1}{3}\\
        a_{3}&=\frac{\left(-1\right)^{3}\left(3+1\right)}{3^{3}}=-\frac{4}{27}\\
        a_{4}&=\frac{\left(-1\right)^{4}\left(4+1\right)}{3^{4}}=\frac{5}{81}
    \end{align*}
\subsection{Пример}
    \textbf{Последовательность Фибоначчи} - рекурентная числовая последовательность. Её отличие от прошлых примеров в том, что она задается не явной формулой, а через предыдущие члены последовательности. Итак,
    \begin{equation*}
        f_1 = 1, \;\;\;\; f_2 = 1, \;\;\;\; f_n = f_{n-1} + f_{n+2}, \;\;\;\; n \geq 3.
    \end{equation*}
    Тогда первые её члены последовательности:
    \begin{equation*}
        \{1,1,2,3,5,8,13,21, \dots\}
    \end{equation*}
    
\newpage
\section{Предел числовой последовательности}
    Рассмотрим последовательность $a_n = \frac{n}{n+1}$. С увеличением $n$ она стремится к 1. Действительно, разницу
    \begin{equation*}
        \left|1 - \frac{n}{n+1}\right| = \left|\frac{1}{n+1}\right| = \frac{n}{n+1}.
    \end{equation*}
    можно сделать настолько маленькой, насколько мы захотим (увеличивая $n$). Будем обозначать это
    \begin{equation*}
        \lim_{n \to \infty} \frac{n}{n+1} = 1.
    \end{equation*}
    В целом, обозначение
    \begin{equation*}
        \lim_{n \to \infty} a_n = A
    \end{equation*}
    означает, что члены последовательности $a_n$ при больших $n$ стремятся к $A$.
\subsection{Определение}
    Последовательность $a_n$ \textbf{имеет предел} $A$ если мы можем сделать члены $a_n$ настолько близкими к $A$ (увеличивая $n$), насколько мы хотим.
    Если этот предел существует, то будем говорить, что последовательность \textbf{сходится} (или является \textbf{сходящейся}). Иначе будем говорить, что последовательность \textbf{расходится}.\\
    \underline{Обозначение:} $\displaystyle\lim_{n \to \infty} a_n = A$ или $a_n \to A$ при $n \to \infty$
    
    \textit{Более формальное определение:}
    \begin{equation*}
        \lim_{n \to \infty} a_n = A
    \end{equation*}
    \begin{equation*}
        \iff
    \end{equation*}
    \begin{equation*}
        \forall \varepsilon > 0 \;\;\; \exists N \in \NN: \forall n > N \implies |a_n - A| < \varepsilon
    \end{equation*}
    \textit{Читается как:} Для любого $\varepsilon$ больше нуля существует натуральное $N$, такое что для любого $n$ больше чем это $N$ следует, что разница между $a_n$ и $A$ по модулю меньше $\varepsilon$. То есть начиная с некоторого $N$ все члены последовательности попадают в $\varepsilon$ окресность точки $A$.
    
\end{document} 